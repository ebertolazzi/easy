\makeatletter
\def\input@path{{../macros/}{../../macros/}}
\makeatother

\documentclass[a4paper,final,11pt]{article}

\usepackage{mydoc}
\newcommand{\package}[1]{\textsc{#1}}
\newcommand{\envname}[1]{\textsf{#1}}
\newcommand{\cmdname}[1]{\texttt{\char`\\#1}}

\author{Enrico Bertolazzi\\ \\
  Laboratorio di Matematica Applicata e Meccanica Computazionale\\
  Dipartimento di Ingegneria Meccanica e Strutturale\\
  Universit\`a degli Studi di Trento\\
  Mesiano, Trento, Italy \\ \\
  \Email{enrico.bertolazzi@ing.unitn.it} }

%%%
%%% HEADER PAGE
%%%
%begin{latexonly}
\usepackage{fancyhdr}
\fancypagestyle{fpage}{
  \fancyhf{} % clear all
  \fancyhead[el]{\fancyplain{}{\sffamily\bfseries\thepage}}
  \fancyhead[er]{\fancyplain{}{\sffamily\bfseries\leftmark}}
  \fancyhead[or]{\fancyplain{}{\sffamily\bfseries\thepage}}
  \fancyhead[ol]{\fancyplain{}{\sffamily\bfseries\rightmark}}
}
\renewcommand{\headrulewidth}{0pt}
%end{latexonly}

%%%
%%% LaTeX2HTML INCLUDES
%%%
\usepackage{html,htmllist,url,verbatim}
\bodytext{BACKGROUND = "bk.jpg"}
%begin{latexonly}
\newcommand{\URL}{\begingroup\urlstyle{sf}\Url}
\newcommand{\Email}{\begingroup\urlstyle{sf}\Url}
%end{latexonly}
\begin{htmlonly}
\newcommand{\URL}[1]{\htmladdnormallink{#1}{#1}}
\end{htmlonly}

%%
%% POSTSCRIPT AND FIGURES
%%
\usepackage{epsfig}
\graphicspath{{:figure:}{figure/}}

%begin{latexonly}
\usepackage{amssymb}
\newcommand{\CIRC}{$\circ$}
\newcommand{\BULLET}{$\bullet$}
%end{latexonly}

\begin{htmlonly}

\newcommand{\EXAMPLE}[1]{
  \begin{BOXED}
    \VRB{#1}
    \input{#1}
  \end{BOXED}
}

\newcommand{\EXAMPLEA}[1]{
  \begin{BOXED}
    \VRB{#1}
    \input{#1}
  \end{BOXED}
}

\newcommand{\EXAMPLESPLIT}[1]{
  \begin{BOXED}
    \VRB{#1.1.tex}
    \VRB{#1.2.tex}
    \input{#1.tex}
  \end{BOXED}
}

\newcommand{\FRAMECODE}[1]{
  \begin{BOXED}
    \input{#1}
  \end{BOXED}
}

\end{htmlonly}

%begin{latexonly}

\usepackage{fancyvrb}

\newdimen\MINISIZE
\newcommand{\VRB}[1]{
  \par
  \VerbatimInput[xleftmargin=\fboxsep,xrightmargin=\fboxsep]{#1}
  \par
}

\newcommand{\FRAMEVRB}[1]{
  \par
  \VerbatimInput[xleftmargin=\fboxsep,
                 xrightmargin=\fboxsep,
                 frame=single]{#1}
  \par
}

\newcommand{\FRAMECODE}[1]{
  \MINISIZE=\textwidth\relax
  \advance\MINISIZE-4\fboxsep\relax
  \begin{center}
  \par
  \fbox{%
  \begin{minipage}{\MINISIZE}
  \input{#1}
  \end{minipage}}
  \par
  \end{center}
}

\newcommand{\EXAMPLE}[1]{
  \FRAMEVRB{#1.tex}
  \FRAMECODE{#1.tex}
}

\newcommand{\EXAMPLEA}[1]{
  \MINISIZE=\textwidth\relax
  \advance\MINISIZE-4\fboxsep\relax
  \begin{center}
  \par
  \fbox{%
  \begin{minipage}{\MINISIZE}
  \vskip\fboxsep%
  \VerbatimInput[xleftmargin=\fboxsep,
                 xrightmargin=\fboxsep,
                 frame=single]{#1.tex}
  \input{#1.tex}
  \end{minipage}}
  \par
  \end{center}
}

\let\EXAMPLESPLIT\EXAMPLE

%end{latexonly}

\usepackage{easymat}%
\usepackage{easybmat}%
\usepackage[definevectors]{easyvector}
\usepackage[math]{easyeqn}

\title{The package \package{easyvector}}

\begin{document}
\maketitle
\begin{abstract}
The \package{easyvector} package is a simple
macro package that provides a C-like syntax for writing vectors or matrices.
\end{abstract}

\tableofcontents
%begin{latexonly}
\clearpage
%end{latexonly}
\pagestyle{fpage}

\def\sectionmark#1{\markboth{The package \package{easyvector}}{The package \package{easyvector}}}
\let\chaptermark\sectionmark
\let\subsectionmark\sectionmark

\section{Some examples with \package{easyvector}}
%%
The package is loaded by means of the usual way:
%%
\begin{verbatim}

  \documentclass{article}
  .
  .
  \usepackage[spacesep,definevectors]{easyvector}
  .
  .

\end{verbatim}
%%
The package option \texttt{spacesep} means that the separator for the
indices is the command \cmdname{smallspace} instead of
``\verb|,|''~(comma).

The package option \texttt{definevectors} means that the command
\cmdname{aa},\ldots, \cmdname{zz} and \cmdname{AA},\ldots,
\cmdname{ZZ} are predefined as vectors.  It also defines the commands
\cmdname{Balpha}, \cmdname{Bbeta} and so on, as bold greek vectors. 
The latex commands \cmdname{aa}, \cmdname{AA}, \cmdname{gg},
\cmdname{ll}, \cmdname{ss}, \cmdname{SS}, \cmdname{tt} are saved in
the commands \cmdname{oldxx} where \verb'xx' is the name of the old
command.

\section{Use of the \cmdname{newvector} command}
%%
The general syntax of \cmdname{newvector} command is
%%
\begin{verbatim}

  \newvector[\cmda,\cmdb]{cmd}

\end{verbatim}
%%
or
%%
\begin{verbatim}

  \newvector(a)[cmd]

\end{verbatim}
%%
In the first case, it creates the new command (macro) \cmdname{cmd}
which executes \cmdname{cmda} when in scalar mode and \cmdname{cmdb}
when in vector mode.  In the second case it creates a new command
\cmdname{cmd} which substitutes the letter \cmdname{mathit}\verb'{a}'
when in scalar, mode and \cmdname{mathbf}\verb'{a}' when in vector
mode.  Scalar mode is activated when \cmdname{cmd} is immediately
followed by~\verb|[|.  In scalar mode everything between
\verb|[|~and~\verb|]| (with balancing) is assumed to be as an index.
For example the commands
%%
\EXAMPLEA{docvector.1}
%
The structure of the \verb|[...]| command is the following
%%
\begin{verbatim}

  [i,j,...,k;x,y,...,z]

\end{verbatim}
%%
where \verb|i,j,...,k| are subscripts and \verb|x,y,...,z| are
superscripts.  The comma ``\verb|,|''~is used as a separator
between different indices, and the semi-colon ``\verb|;|''~separates
subscripts and superscripts.  There are no limits on the number of
indices, and the code is reentrant, as the following example
illustrates
%%
\EXAMPLEA{docvector.2}
%%

\section{Use of the \texttt{!} command}
%%
It is possible to enforce vector mode also when using indices
by using the character~\verb'!' before~\verb'['
%%
\EXAMPLEA{docvector.3}
%%

\section{Use of the \texttt{\char`\\newcustomvector} command}
%%
In some circumstances the command \cmdname{newcustomvector} can be
useful.  Is is essentially the \cmdname{newvector} command with an
extra argument that is a macro to manage the index part.
%%
\EXAMPLEA{docvector.4}
%%

\textbf{Important:} For old users (version $< 0.6$) the command
\cmdname{customindex} is suppressed and the \cmdname{newcustomvector}
is used instead.

\section{The ``definevectors'' option}
%%
This option defines the following vectors for you:
%%
\begin{verbatim}

  \aa,\bb,...,\zz    \AA,\BB,...,\ZZ
  \Balpha, \Bbeta, ..., \Bomega

\end{verbatim}
%%
for example
%%
\EXAMPLEA{docvector.5}
%%

\section{The ``@'' convention}
%%
In linear algebra it is common to use the notation $\AA[@,j]$ to
denote the vector formed by the $j^{th}$ column of $\AA$.
%%
Note that $\AA$ is in vector format not in scalar format ($\AA[]$).
We can use ``$\bullet$'' 
as an index in a vector forcing the vector mode by using @ as follows: 
%%
\EXAMPLEA{docvector.6}
%%
\end{document}
