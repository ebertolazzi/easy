\makeatletter
\def\input@path{{../macros/}{../../macros/}}
\makeatother


\documentclass[a4paper]{article}

\usepackage{mydoc}
\newcommand{\package}[1]{\textsc{#1}}
\newcommand{\envname}[1]{\textsf{#1}}
\newcommand{\cmdname}[1]{\texttt{\char`\\#1}}

\author{Enrico Bertolazzi\\ \\
  Laboratorio di Matematica Applicata e Meccanica Computazionale\\
  Dipartimento di Ingegneria Meccanica e Strutturale\\
  Universit\`a degli Studi di Trento\\
  Mesiano, Trento, Italy \\ \\
  \Email{enrico.bertolazzi@ing.unitn.it} }

%%%
%%% HEADER PAGE
%%%
%begin{latexonly}
\usepackage{fancyhdr}
\fancypagestyle{fpage}{
  \fancyhf{} % clear all
  \fancyhead[el]{\fancyplain{}{\sffamily\bfseries\thepage}}
  \fancyhead[er]{\fancyplain{}{\sffamily\bfseries\leftmark}}
  \fancyhead[or]{\fancyplain{}{\sffamily\bfseries\thepage}}
  \fancyhead[ol]{\fancyplain{}{\sffamily\bfseries\rightmark}}
}
\renewcommand{\headrulewidth}{0pt}
%end{latexonly}

%%%
%%% LaTeX2HTML INCLUDES
%%%
\usepackage{html,htmllist,url,verbatim}
\bodytext{BACKGROUND = "bk.jpg"}
%begin{latexonly}
\newcommand{\URL}{\begingroup\urlstyle{sf}\Url}
\newcommand{\Email}{\begingroup\urlstyle{sf}\Url}
%end{latexonly}
\begin{htmlonly}
\newcommand{\URL}[1]{\htmladdnormallink{#1}{#1}}
\end{htmlonly}

%%
%% POSTSCRIPT AND FIGURES
%%
\usepackage{epsfig}
\graphicspath{{:figure:}{figure/}}

%begin{latexonly}
\usepackage{amssymb}
\newcommand{\CIRC}{$\circ$}
\newcommand{\BULLET}{$\bullet$}
%end{latexonly}

\begin{htmlonly}

\newcommand{\EXAMPLE}[1]{
  \begin{BOXED}
    \VRB{#1}
    \input{#1}
  \end{BOXED}
}

\newcommand{\EXAMPLEA}[1]{
  \begin{BOXED}
    \VRB{#1}
    \input{#1}
  \end{BOXED}
}

\newcommand{\EXAMPLESPLIT}[1]{
  \begin{BOXED}
    \VRB{#1.1.tex}
    \VRB{#1.2.tex}
    \input{#1.tex}
  \end{BOXED}
}

\newcommand{\FRAMECODE}[1]{
  \begin{BOXED}
    \input{#1}
  \end{BOXED}
}

\end{htmlonly}

%begin{latexonly}

\usepackage{fancyvrb}

\newdimen\MINISIZE
\newcommand{\VRB}[1]{
  \par
  \VerbatimInput[xleftmargin=\fboxsep,xrightmargin=\fboxsep]{#1}
  \par
}

\newcommand{\FRAMEVRB}[1]{
  \par
  \VerbatimInput[xleftmargin=\fboxsep,
                 xrightmargin=\fboxsep,
                 frame=single]{#1}
  \par
}

\newcommand{\FRAMECODE}[1]{
  \MINISIZE=\textwidth\relax
  \advance\MINISIZE-4\fboxsep\relax
  \begin{center}
  \par
  \fbox{%
  \begin{minipage}{\MINISIZE}
  \input{#1}
  \end{minipage}}
  \par
  \end{center}
}

\newcommand{\EXAMPLE}[1]{
  \FRAMEVRB{#1.tex}
  \FRAMECODE{#1.tex}
}

\newcommand{\EXAMPLEA}[1]{
  \MINISIZE=\textwidth\relax
  \advance\MINISIZE-4\fboxsep\relax
  \begin{center}
  \par
  \fbox{%
  \begin{minipage}{\MINISIZE}
  \vskip\fboxsep%
  \VerbatimInput[xleftmargin=\fboxsep,
                 xrightmargin=\fboxsep,
                 frame=single]{#1.tex}
  \input{#1.tex}
  \end{minipage}}
  \par
  \end{center}
}

\let\EXAMPLESPLIT\EXAMPLE

%end{latexonly}

\usepackage{easymat}%
\usepackage{easybmat}%
\usepackage[definevectors]{easyvector}
\usepackage[math]{easyeqn}

\title{The package \package{easyeqn}}

\begin{document}
\maketitle
\begin{abstract}
  The package \package{easyeqn} introduces some equation environments
  that simplify writing of equations.  It uses a syntax similar to the
  array environment to define the column alignment.  The label field
  is fully customizable.  A package option permits to number only
  those equations that were \emph{labeled and referenced}.  Some
  additional macros are also included to facilitate typing of
  formulae.
%   The option ``showkeys'' modifies the \cmdname{label},
%   \cmdname{eqref} and \cmdname{refeq} commands so that the `internal'
%   key is printed.
\end{abstract}

\tableofcontents
%begin{latexonly}
\clearpage
%end{latexonly}

\pagestyle{fpage}
\def\sectionmark#1{\markboth{The package \package{easyeqn}}{The package \package{easyeqn}}}
\let\chaptermark\sectionmark
\let\subsectionmark\sectionmark

\section{Some examples with \package{easyeqn}}
Load the package using the usual syntax:
%
\VRB{doceqn.1.tex}
%
The package\footnote{the option ``showkeys'' is eliminated in this
version because now \package{easyeqn} is compatible with 
\package{showkeys} package}
introduces the \envname{EQ} and \envname{EQA}~environments.
The package options are:
\begin{dotlist}
%%
\item[allnumber]
     Means that all of the \envname{EQ} and \envname{EQA}
     environments are numbered. Without that option, only
     those \envname{EQ} and \envname{EQA} environment that
     are \emph{labeled} and \emph{referenced} are numbered.
%%
\item[warning]
     Causes equations that are labeled but \emph{not referenced}
     to be flagged.
%%
\item[easyold]
     Produce obsolete environment \envname{EQS}, \envname{EQS*},
     \envname{EQ*}, \envname{EQA*} for backward compatibilty.
%%
\item[fleqn]
     Flushes equation on the left margin.
%%
\item[leqno]
     Writes equation number on the left.
%%
\item[math]
     Define additional macros for mathematics.
%%
\end{dotlist}
 


\section{Use of the \envname{EQ} environment}
Look at this example:
%
\EXAMPLEA{doceqn.2}
%
Note that the reference is done by \cmdname{eqref} or \cmdname{refeq}. 
The command \cmdname{refeq} produces the same output as \cmdname{ref},
while \cmdname{eqref} uses \texttt{( )} for the output.

\textbf{Remark:} Due to the algorithm implementation,
in order to obtain the right cross reference, you need
to recompile the file 3~times.  The use
\cmdname{label} is not permitted inside \envname{EQ*} environments and
if you use \cmdname{ref} for referencing
equations the results are unpredictable\footnote{%
The previous release to label an equation uses the command
\texttt{\\eqlabel}, for backward compatibility this command
is maintained but the user should use the \texttt{\\label}
command
}.

This is another example:
%
\EXAMPLEA{doceqn.3}
%
Note that between \verb|[...]| you can specify the column alignment
in the same way as in the \envname{array} or \envname{tabular} 
environment\footnote{In a previous release of \package{easyeqn}
multicolumn alignment was implemented in a \envname{EQS} environment.
However to maintain backward compatibility such an environment
are maintained}.
The permitted alignment are \verb|l| for left alignment,
\verb|r| for right alignment and \verb|c| for centering. There is also
the character ``\verb'.''' that used between definition of two columns
disable the spacing between columns as in the following example taken 
from the documentation of \package{eqnarray} of Roland Winkler;
%
\EXAMPLEA{doceqn.4}
%
in this example we have used also the command \cmdname{eqmulticol} the 
syntax is the following
\begin{verbatim}
   \eqmulticol{ncol}{align}{body}
\end{verbatim}
where
\begin{dotlist}
  \item[ncol] is the number of column to fuse.
  \item[aling] is the alignment, it can be \verb'l',  \verb'r',  \verb'c'.
  \item[body] is the thing to put across the column.
\end{dotlist}

\section{The \texttt{\char`\\yesnumber} command}
If you want to number an equation without reference it, use the
\cmdname{yesnumber} command as this example shows:
%
\EXAMPLEA{doceqn.5}
%




\section{Use of \envname{EQA} environment}
%
\EXAMPLEA{doceqn.6}
%
Note that only the referenced lines or the lines with 
\cmdname{yesnumber} are numbered.


\section{The \texttt{\char`\\label} command}
It is possible to use custom label by \cmdname{label} command. 
The syntax is one of the following:
\begin{verbatim}
  \label{labelname}
  \label[eqnum]
  \label[eqnum]{labelname}
  \label(eqnum)
  \label(eqnum){labelname}
\end{verbatim}
where ``\verb'[eqnum]''' is an optional argument that if defined,
equation displays ``\verb'eqnum''' instead of
``\verb'(equation number)'''.  The equation counter is not advanced and
``\verb'labelname''' if present will refer to ``\verb'eqnum'''.

If we use where ``\verb'(eqnum)''' is an optional argument that if
defined, equation displays ``\verb'(eqnum)''' instead of
``\verb'(equation number)'''.  The equation counter is not advanced
and ``\verb'labelname''' if present will refer to ``\verb'eqnum'''.

Look at this example:
%
\EXAMPLEA{doceqn.7}
%
Notice that custom label are always displayed also if they are
not referenced.

\section{Label positioning}
It is possible to change the default position os a single 
label by the commands:
\begin{dotlist}
  \item\cmdname{eqlabeltop}
  \item\cmdname{eqlabelbot}
  \item\cmdname{eqlabelcenter}
\end{dotlist}
For example:
%
\EXAMPLEA{doceqn.8}
%


\section{Sub-numbering}
To sub-number equation instead of use something like
%
\begin{verbatim}
  \begin{subequations}
  \begin{EQ}...
    
  \end{EQ}
  \end{subequations}
\end{verbatim}
%
I prefer to use \cmdname{label} command with the character \verb'~'
as a shortcut for the command \cmdname{theequation}. The following example shows 
the use
%
\EXAMPLEA{doceqn.9}
%
the \cmdname{yesnumber} command is necessary to enforce advancing of 
equation counter.


\section{Use of \envname{fleqn} and \envname{leqno} option}
You can use \envname{fleqn} to flux left the equations or
\envname{leqno} to number equations on the left.
If for example we have used
\begin{verbatim}

  \documentclass{article}
  .
  .
  \usepackage[fleqn,leqno]{easyeqn}
  .
  .

\end{verbatim}
the effect with this example:%
%
\equationleft\numberleft 
\EXAMPLEA{doceqn.10}
\equationcenter\numberright
%
The same effect can be obtained everywhere using the commands
\cmdname{equationleft} and \cmdname{numberleft} before defining
the equation. To restore the default values use the commands
\cmdname{equationcenter} and \cmdname{numberright} after the equation.
%

\section{Cosmetic changes}
It is possible to modify a little the appearance of the equations.
There are three parameter that it is possible to change
\begin{dotlist}
  \item[left indent] When the equations are fluxed left, the left indent
  can be changed by the command \cmdname{eqleftmargin}.
\begin{verbatim}
   
   \eqleftmargin{new indent}
	
\end{verbatim}
%    
for example
%
\begin{verbatim}
  
  \eqleftmargin{1cm}

\end{verbatim}
The default value for the left margin is \cmdname{leftmargini}. 

\item[equation spacing] The spacing around a formula,
(default \verb'7pt') can be changed by the command
\begin{verbatim}

  \eqspacing{new spacing}

\end{verbatim}    
for example
\begin{verbatim}
  
  \eqspacing{4pt}

\end{verbatim}
\item[column spacing]
The spacing between columns (default value \verb'4pt') can be changed by the command
\begin{verbatim}

  \eqcolumnsep{new spacing}

\end{verbatim}  
for example
\begin{verbatim}
    
  \eqcolumnsep{10pt}

\end{verbatim}
\item[row spacing] The spacing between rows in multiple
equations (default value \verb'7pt') can be changed by the command
\begin{verbatim}

  \eqrowsep{new spacing}

\end{verbatim}
for example
\begin{verbatim}

  \eqrowsep{10pt}

\end{verbatim}
\end{dotlist}

for example
%
\EXAMPLEA{doceqn.11}
%


\section{Additional macros}
Using the package as follow
\begin{verbatim}
  \documentclass{article}
  .
  .
  \usepackage[...,math]{easyeqn}
  .
  .
\end{verbatim}
%
we can use additional macros for typesetting mathematics, the 
following macros are defined  \cmdname{frac}, \cmdname{dfrac},
\cmdname{tfrac}, \cmdname{binom} and  \cmdname{boxed} which use is described 
in the following example:
%
\EXAMPLEA{doceqn.12}
%
The macro \cmdname{eqbox} is defined, the effect is the following
%
\EXAMPLEA{doceqn.13}
%
The macros \cmdname{norm} and \cmdname{abs} are defined,
the effect is the following
%
\EXAMPLEA{doceqn.14}
%
The macro \cmdname{ParDer} is defined, the effect is the following
%
\EXAMPLEA{doceqn.15}
%
notice that single item of derivative must be a single letter (or a 
macro) or must be inside a group \verb'{ ... }'.
If you use \cmdname{ParDer} with package \package{easyvector}
remember to put macros in brace when use ``$[$ $]$'' as follows:
%
\EXAMPLEA{doceqn.16}
%
otherwise you obtain strange think like the following
%
\EXAMPLEA{doceqn.17}
%
The macros \cmdname{DIV}, \cmdname{GRAD} and \cmdname{LAPLA} are defined,
the effect is the following
%
\EXAMPLEA{doceqn.18}
%
The macro \cmdname{SUM} is defined, the effect is the following
%
\EXAMPLEA{doceqn.19}
%
The macro \cmdname{PROD} is defined, the effect is the following
%
\EXAMPLEA{doceqn.20}
%
The environment \envname{ARRAY} is defined, is a simple subset of the 
environment \envname{array} with a different spacing; look the following 
example
%
\EXAMPLEA{doceqn.21}
%
The environment \envname{MATRIX} is defined, is a simple replacement
of \cmdname{matrix} command with a different spacing; look the following 
example
%
\EXAMPLEA{doceqn.22}
%

\end{document}
