\makeatletter
\def\input@path{{../macros/}{../../macros/}}
\makeatother


\documentclass[a4paper,final,11pt]{article}
% \usepackage{showkeys}

\usepackage{mydoc}
\newcommand{\package}[1]{\textsc{#1}}
\newcommand{\envname}[1]{\textsf{#1}}
\newcommand{\cmdname}[1]{\texttt{\char`\\#1}}

\author{Enrico Bertolazzi\\ \\
  Laboratorio di Matematica Applicata e Meccanica Computazionale\\
  Dipartimento di Ingegneria Meccanica e Strutturale\\
  Universit\`a degli Studi di Trento\\
  Mesiano, Trento, Italy \\ \\
  \Email{enrico.bertolazzi@ing.unitn.it} }

%%%
%%% HEADER PAGE
%%%
%begin{latexonly}
\usepackage{fancyhdr}
\fancypagestyle{fpage}{
  \fancyhf{} % clear all
  \fancyhead[el]{\fancyplain{}{\sffamily\bfseries\thepage}}
  \fancyhead[er]{\fancyplain{}{\sffamily\bfseries\leftmark}}
  \fancyhead[or]{\fancyplain{}{\sffamily\bfseries\thepage}}
  \fancyhead[ol]{\fancyplain{}{\sffamily\bfseries\rightmark}}
}
\renewcommand{\headrulewidth}{0pt}
%end{latexonly}

%%%
%%% LaTeX2HTML INCLUDES
%%%
\usepackage{html,htmllist,url,verbatim}
\bodytext{BACKGROUND = "bk.jpg"}
%begin{latexonly}
\newcommand{\URL}{\begingroup\urlstyle{sf}\Url}
\newcommand{\Email}{\begingroup\urlstyle{sf}\Url}
%end{latexonly}
\begin{htmlonly}
\newcommand{\URL}[1]{\htmladdnormallink{#1}{#1}}
\end{htmlonly}

%%
%% POSTSCRIPT AND FIGURES
%%
\usepackage{epsfig}
\graphicspath{{:figure:}{figure/}}

%begin{latexonly}
\usepackage{amssymb}
\newcommand{\CIRC}{$\circ$}
\newcommand{\BULLET}{$\bullet$}
%end{latexonly}

\begin{htmlonly}

\newcommand{\EXAMPLE}[1]{
  \begin{BOXED}
    \VRB{#1}
    \input{#1}
  \end{BOXED}
}

\newcommand{\EXAMPLEA}[1]{
  \begin{BOXED}
    \VRB{#1}
    \input{#1}
  \end{BOXED}
}

\newcommand{\EXAMPLESPLIT}[1]{
  \begin{BOXED}
    \VRB{#1.1.tex}
    \VRB{#1.2.tex}
    \input{#1.tex}
  \end{BOXED}
}

\newcommand{\FRAMECODE}[1]{
  \begin{BOXED}
    \input{#1}
  \end{BOXED}
}

\end{htmlonly}

%begin{latexonly}

\usepackage{fancyvrb}

\newdimen\MINISIZE
\newcommand{\VRB}[1]{
  \par
  \VerbatimInput[xleftmargin=\fboxsep,xrightmargin=\fboxsep]{#1}
  \par
}

\newcommand{\FRAMEVRB}[1]{
  \par
  \VerbatimInput[xleftmargin=\fboxsep,
                 xrightmargin=\fboxsep,
                 frame=single]{#1}
  \par
}

\newcommand{\FRAMECODE}[1]{
  \MINISIZE=\textwidth\relax
  \advance\MINISIZE-4\fboxsep\relax
  \begin{center}
  \par
  \fbox{%
  \begin{minipage}{\MINISIZE}
  \input{#1}
  \end{minipage}}
  \par
  \end{center}
}

\newcommand{\EXAMPLE}[1]{
  \FRAMEVRB{#1.tex}
  \FRAMECODE{#1.tex}
}

\newcommand{\EXAMPLEA}[1]{
  \MINISIZE=\textwidth\relax
  \advance\MINISIZE-4\fboxsep\relax
  \begin{center}
  \par
  \fbox{%
  \begin{minipage}{\MINISIZE}
  \vskip\fboxsep%
  \VerbatimInput[xleftmargin=\fboxsep,
                 xrightmargin=\fboxsep,
                 frame=single]{#1.tex}
  \input{#1.tex}
  \end{minipage}}
  \par
  \end{center}
}

\let\EXAMPLESPLIT\EXAMPLE

%end{latexonly}

\usepackage{easymat}%
\usepackage{easybmat}%
\usepackage[definevectors]{easyvector}
\usepackage[math]{easyeqn}

\title{The package \package{easyeqn}}

\begin{document}
\maketitle
\begin{abstract}
    The package \package{easyeqn} introduces some equation
    environments that simplify the typesetting of equations.  It uses
    a syntax similar to the array environment to define the column
    alignment.  The label field is fully customizable.  A package
    option permits to number only those equations that were
    \emph{labeled and referenced}.i Additional macros are also
    included to facilitate the typing of formulae.
\end{abstract}

\tableofcontents
%begin{latexonly}
\clearpage
%end{latexonly}

\pagestyle{fpage}
\def\sectionmark#1{\markboth{The package \package{easyeqn}}{The package \package{easyeqn}}}
\let\chaptermark\sectionmark
\let\subsectionmark\sectionmark

\section{Some examples with \package{easyeqn}}
The package is loaded by means of the usual syntax:
%%
\VRB{doceqn.1.tex}
%%
The package\footnote{the option ``showkeys'' is eliminated
because the new release of \package{easyeqn} is compatible with 
the \package{showkeys} package}
introduces the \envname{EQ} and \envname{EQA}~environments.
The package options are:
\begin{dotlist}
%%
\item[\textbf{allnumber}]
     Means that all of the \envname{EQ} and \envname{EQA}
     environments are numbered. Without that option, only
     those \envname{EQ} and \envname{EQA} environments that
     are explicitly \emph{labeled} and \emph{referenced} are numbered.
%%
\item[\textbf{warning}]
     Causes the flagging of the equations that are labeled but
     \emph{not referenced}.
%%
\item[\textbf{easyold}]
     Produces obsolete environment \envname{EQS}, \envname{EQS*},
     \envname{EQ*}, \envname{EQA*} for backward compatibilty.
%%
\item[\textbf{fleqn}]
     equations will be left-justify.
%%
\item[\textbf{leqno}]
     Writes equation number on the left.
%%
\item[\textbf{math}]
     Defines additional macros for mathematics.
%%
\end{dotlist}
 


\section{Use of the \envname{EQ} environment}
%%
The use of \envname{EQ} environment is best unserstood by the following
example:
%%
\EXAMPLEA{doceqn.2}
%%
Note that the reference is done by \cmdname{eqref} or \cmdname{refeq}. 
The command \cmdname{refeq} produces the same output as \cmdname{ref},
while \cmdname{eqref} uses \texttt{( )} for the output.

\textbf{Remark:} Due to the algorithm implementation,
in order to obtain the right cross reference, you need
to recompile the file 3~times.  The use of
\cmdname{label} is not permitted from within \envname{EQ*} environments.
If you use \cmdname{ref} to reference equations results
are unpredictable%
%%%%%%%%%%%%%%%%%%%%%%%%%%%%%%%%%%%%%%%%%%%%%%%%%%%%%%%%%%%%%%%
\footnote{%
The previous release used the command
\texttt{\\eqlabel} for equation labelling,
for backward compatibility this command
is maintained but the user should use the \texttt{\\label}
command
}.

Here is another example:
%%
\EXAMPLEA{doceqn.3}
%%
Note that between \verb|[...]| you can specify the column alignment
in the same way as in the \envname{array} or \envname{tabular} 
environment%
%%%%%%%%%%%%%%%%%%%%%%%%%%%%%%%%%%%%%%%%%%%%%%%%%%%%%%%%%%%%%%%%%%
\footnote{In a previous release of \package{easyeqn}
multicolumn alignment was implemented in a \envname{EQS} environment.
However to keep backward compatibility such an environment
is maintained}.
%%%%%%%%%%%%%%%%%%%%%%%%%%%%%%%%%%%%%%%%%%%%%%%%%%%%%%%%%%%%%%%%%%
The permitted alignment are \verb|l| for left alignment,
\verb|r| for right alignment and \verb|c| for centering. There is also
the character ``\verb'.''' that if used between the definition of two columns,
disables the spacing between columns as in the following example, which is
taken from the documentation of \package{eqnarray} of Roland Winkler;
%%
\EXAMPLEA{doceqn.4}
%%
In the above example the command \cmdname{eqmulticol} has been introduced.
Its syntax is:
%%
\begin{verbatim}
   \eqmulticol{ncol}{align}{body}
\end{verbatim}
%%
where:
%%
\begin{dotlist}
  \item[\textbf{ncol}]  number of column to merge.
  \item[\textbf{aling}] alignment, parameter to be chosen among the set 
                        \verb'l',  \verb'r',  \verb'c'.
  \item[\textbf{body}] expression to put across the column.
\end{dotlist}

\section{The \texttt{\char`\\yesnumber} command}
%%
If may you want to number an equation without reference it.
The \cmdname{yesnumber} command does the work as this example shows:
%%
\EXAMPLEA{doceqn.5}
%%




\section{Use of \envname{EQA} environment}
%%
\EXAMPLEA{doceqn.6}
%%
Note that only the referenced lines or the lines with 
\cmdname{yesnumber} are numbered.


\section{The \texttt{\char`\\label} command}
It is possible to use custom label by \cmdname{label} command. 
The syntax is one of the following:
%%
\begin{verbatim}
  \label{labelname}
  \label[eqnum]
  \label[eqnum]{labelname}
  \label(eqnum)
  \label(eqnum){labelname}
\end{verbatim}
%%
where \verb|[eqnum]| is an optional argument that if defined,
causes the equation displays \verb|eqnum| instead of
\verb'(equation number)'.  The equation counter is not advanced and
\verb'labelname' if present will refer to \verb'eqnum'.

For example:
%%
\EXAMPLEA{doceqn.7}
%%
Note that custom label are always displayed even if
not referenced.

\section{Label positioning}
It is possible to change the default position of a single 
label by the commands:
%%
\begin{dotlist}
  \item\cmdname{eqlabeltop}
  \item\cmdname{eqlabelbot}
  \item\cmdname{eqlabelcenter}
\end{dotlist}
%%
For example:
%%
\EXAMPLEA{doceqn.8}
%%


\section{Sub-numbering}
%%
To sub-number equation, instead of use something like
%%
\begin{verbatim}
  \begin{subequations}
  \begin{EQ}...
    
  \end{EQ}
  \end{subequations}
\end{verbatim}
%%
I prefer to use the \cmdname{label} command with the character \verb'~'
as a shortcut for the command \cmdname{theequation}. The following example shows 
the use:
%%
\EXAMPLEA{doceqn.9}
%%
the \cmdname{yesnumber} command is necessary to enforce the advancing of 
equation counter.


\section{Use of \envname{fleqn} and \envname{leqno} option}
%%
You can use \envname{fleqn} to left justify the equations or
\envname{leqno} to number equations on the left.
For example:
%%
\begin{verbatim}

  \documentclass{article}
  .
  .
  \usepackage[fleqn,leqno]{easyeqn}
  .
  .

\end{verbatim}
%%
and the following example shows the effect
%%
\equationleft\numberleft 
\EXAMPLEA{doceqn.10}
\equationcenter\numberright
%
The same effect can be obtained everywhere using the commands
\cmdname{equationleft} and \cmdname{numberleft} before defining
the equation. To restore the default values use the commands
\cmdname{equationcenter} and \cmdname{numberright} after the equation.
%

\section{Cosmetic changes}
%%
It is possible to slighly modify the appearance of the equations.
There are three parameters that can be changed:
%%
\begin{dotlist}
  \item[\textbf{left indent}] Whenever equations are left justified, the left indent
  can be changed by the command \cmdname{eqleftmargin}.
%%
\begin{verbatim}
   
   \eqleftmargin{new indent}
	
\end{verbatim}
%%   
for example
%%
\begin{verbatim}
  
  \eqleftmargin{1cm}

\end{verbatim}
%%
The default value for the left margin is \cmdname{leftmargini}. 

\item[\textbf{equation spacing}] The spacing of a formula,
(default \verb'7pt') can be controlled by the command
%%
\begin{verbatim}

  \eqspacing{new spacing}

\end{verbatim}    
%%
for example
%%
\begin{verbatim}
  
  \eqspacing{4pt}

\end{verbatim}
%%
\item[\textbf{column spacing}]
The spacing among columns (default value \verb'4pt') can be changed by the command
%%
\begin{verbatim}

  \eqcolumnsep{new spacing}

\end{verbatim}  
%%
for example
%%
\begin{verbatim}
    
  \eqcolumnsep{10pt}

\end{verbatim}
%%
\item[\textbf{row spacing}] The spacing among rows in multiple
equations (default value \verb'7pt') can be changed by the command
%%
\begin{verbatim}

  \eqrowsep{new spacing}

\end{verbatim}
%%
for example
%%
\begin{verbatim}

  \eqrowsep{10pt}

\end{verbatim}
%%
\end{dotlist}
%%
for example
%%
\EXAMPLEA{doceqn.11}
%%


\section{Additional macros}
%%
Using the package as follow
%%
\begin{verbatim}
  \documentclass{article}
  .
  .
  \usepackage[...,math]{easyeqn}
  .
  .
\end{verbatim}
%%
as additional macros useful for typesetting mathematics
can be invoked. The macros are defined as \cmdname{frac}, \cmdname{dfrac},
\cmdname{tfrac}, \cmdname{binom} and  \cmdname{boxed} and their use is described 
in the following example:
%%
\EXAMPLEA{doceqn.12}
%%
Definition of the macro \cmdname{eqbox} and its effect:
%%
\EXAMPLEA{doceqn.13}
%%
Definition of the macros \cmdname{norm} and \cmdname{abs}
and their effect:
%
\EXAMPLEA{doceqn.14}
%%
Definition of the macro \cmdname{ParDer} and its effect:
%%
\EXAMPLEA{doceqn.15}
%%
Notice the single item of the derivatives must be a single letter (or a 
macro) or must be inside a group \verb'{ ... }'.
If you use \cmdname{ParDer} with package \package{easyvector}
remember to put macros in brace when use ``$[$ $]$'' as follows:
%%
\EXAMPLEA{doceqn.16}
%%
otherwise you obtain weird results like the following
%%
\EXAMPLEA{doceqn.17}
%%
Definition of the macros \cmdname{DIV}, \cmdname{GRAD} and \cmdname{LAPLA}
and their effect:
%%
\EXAMPLEA{doceqn.18}
%%
Definition of the macro \cmdname{SUM} and its effect:
%%
\EXAMPLEA{doceqn.19}
%%
Definition of the macro \cmdname{PROD} and its effect:
%%
\EXAMPLEA{doceqn.20}
%%
The environment \envname{ARRAY} is defined, is a simple subset of the 
environment \envname{array} with a different spacing; look the following 
example
%%
\EXAMPLEA{doceqn.21}
%%
The environment \envname{MATRIX} is defined, is a simple replacement
of \cmdname{matrix} command with a different spacing; look the following 
example
%%
\EXAMPLEA{doceqn.22}
%%

\end{document}

\clearpage
\eqspacing{20pt}
\eqlabeltop

aaa aaa aaa aaa aaa aaa aaa aaa aaa aaa
aaa aaa aaa aaa aaa aaa aaa aaa aaa aaa
aaa aaa aaa aaa aaa aaa aaa aaa aaa aaa
aaa aaa aaa aaa aaa aaa aaa aaa aaa aaa
\begin{EQ}\yesnumber
    1 =1
\end{EQ}
aaa aaa aaa aaa aaa aaa aaa aaa aaa aaa
aaa aaa aaa aaa aaa aaa aaa aaa aaa aaa
aaa aaa aaa aaa aaa aaa aaa aaa aaa aaa
\begin{EQ}\yesnumber
    1 =1 \\
    1 =1
\end{EQ}
aaa aaa aaa aaa aaa aaa aaa aaa aaa aaa
aaa aaa aaa aaa aaa aaa aaa aaa aaa aaa
aaa aaa aaa aaa aaa aaa aaa aaa aaa aaa
\begin{EQA}[c]\yesnumber
    1 =1
\end{EQA}
aaa aaa aaa aaa aaa aaa aaa aaa aaa aaa
aaa aaa aaa aaa aaa aaa aaa aaa aaa aaa
aaa aaa aaa aaa aaa aaa aaa aaa aaa aaa
\begin{EQA}[c]\yesnumber
    1 =1 \\
    1 =1
\end{EQA}
aaa aaa aaa aaa aaa aaa aaa aaa aaa aaa
aaa aaa aaa aaa aaa aaa aaa aaa aaa aaa
aaa aaa aaa aaa aaa aaa aaa aaa aaa aaa

\end{document}
