\makeatletter
\def\input@path{{../macros/}{../../macros/}}
\makeatother

\documentclass[a4paper,final,11pt]{article}

\usepackage{mydoc}
\newcommand{\package}[1]{\textsc{#1}}
\newcommand{\envname}[1]{\textsf{#1}}
\newcommand{\cmdname}[1]{\texttt{\char`\\#1}}

\author{Enrico Bertolazzi\\ \\
  Laboratorio di Matematica Applicata e Meccanica Computazionale\\
  Dipartimento di Ingegneria Meccanica e Strutturale\\
  Universit\`a degli Studi di Trento\\
  Mesiano, Trento, Italy \\ \\
  \Email{enrico.bertolazzi@ing.unitn.it} }

%%%
%%% HEADER PAGE
%%%
%begin{latexonly}
\usepackage{fancyhdr}
\fancypagestyle{fpage}{
  \fancyhf{} % clear all
  \fancyhead[el]{\fancyplain{}{\sffamily\bfseries\thepage}}
  \fancyhead[er]{\fancyplain{}{\sffamily\bfseries\leftmark}}
  \fancyhead[or]{\fancyplain{}{\sffamily\bfseries\thepage}}
  \fancyhead[ol]{\fancyplain{}{\sffamily\bfseries\rightmark}}
}
\renewcommand{\headrulewidth}{0pt}
%end{latexonly}

%%%
%%% LaTeX2HTML INCLUDES
%%%
\usepackage{html,htmllist,url,verbatim}
\bodytext{BACKGROUND = "bk.jpg"}
%begin{latexonly}
\newcommand{\URL}{\begingroup\urlstyle{sf}\Url}
\newcommand{\Email}{\begingroup\urlstyle{sf}\Url}
%end{latexonly}
\begin{htmlonly}
\newcommand{\URL}[1]{\htmladdnormallink{#1}{#1}}
\end{htmlonly}

%%
%% POSTSCRIPT AND FIGURES
%%
\usepackage{epsfig}
\graphicspath{{:figure:}{figure/}}

%begin{latexonly}
\usepackage{amssymb}
\newcommand{\CIRC}{$\circ$}
\newcommand{\BULLET}{$\bullet$}
%end{latexonly}

\begin{htmlonly}

\newcommand{\EXAMPLE}[1]{
  \begin{BOXED}
    \VRB{#1}
    \input{#1}
  \end{BOXED}
}

\newcommand{\EXAMPLEA}[1]{
  \begin{BOXED}
    \VRB{#1}
    \input{#1}
  \end{BOXED}
}

\newcommand{\EXAMPLESPLIT}[1]{
  \begin{BOXED}
    \VRB{#1.1.tex}
    \VRB{#1.2.tex}
    \input{#1.tex}
  \end{BOXED}
}

\newcommand{\FRAMECODE}[1]{
  \begin{BOXED}
    \input{#1}
  \end{BOXED}
}

\end{htmlonly}

%begin{latexonly}

\usepackage{fancyvrb}

\newdimen\MINISIZE
\newcommand{\VRB}[1]{
  \par
  \VerbatimInput[xleftmargin=\fboxsep,xrightmargin=\fboxsep]{#1}
  \par
}

\newcommand{\FRAMEVRB}[1]{
  \par
  \VerbatimInput[xleftmargin=\fboxsep,
                 xrightmargin=\fboxsep,
                 frame=single]{#1}
  \par
}

\newcommand{\FRAMECODE}[1]{
  \MINISIZE=\textwidth\relax
  \advance\MINISIZE-4\fboxsep\relax
  \begin{center}
  \par
  \fbox{%
  \begin{minipage}{\MINISIZE}
  \input{#1}
  \end{minipage}}
  \par
  \end{center}
}

\newcommand{\EXAMPLE}[1]{
  \FRAMEVRB{#1.tex}
  \FRAMECODE{#1.tex}
}

\newcommand{\EXAMPLEA}[1]{
  \MINISIZE=\textwidth\relax
  \advance\MINISIZE-4\fboxsep\relax
  \begin{center}
  \par
  \fbox{%
  \begin{minipage}{\MINISIZE}
  \vskip\fboxsep%
  \VerbatimInput[xleftmargin=\fboxsep,
                 xrightmargin=\fboxsep,
                 frame=single]{#1.tex}
  \input{#1.tex}
  \end{minipage}}
  \par
  \end{center}
}

\let\EXAMPLESPLIT\EXAMPLE

%end{latexonly}

\usepackage{easymat}%
\usepackage{easybmat}%
\usepackage[definevectors]{easyvector}
\usepackage[math]{easyeqn}
\usepackage{easytable}

\title{The package \package{easytable}}

\begin{document}
\maketitle
\begin{abstract}
  The \package{easytable} package is a macro package for writing
  tables, with equal column widths or equal rows heights or both, with
  various kinds of rules~(lines) between rows and columns.  It uses an
  array/tabular-like syntax.
\end{abstract}

\pagestyle{fpage}
\def\sectionmark#1{\markboth{The package \package{easytable}}{The package \package{easytable}}}
\let\chaptermark\sectionmark
\let\subsectionmark\sectionmark

\section{How to use it}
%%
The package is loaded by means of the usual way:
%%
\begin{verbatim}
  \documentclass{article}
  .
  .
  \usepackage[thinlines,thicklines]{easytab}
  .
  .
\end{verbatim}
%%
The options \texttt{thinlines}, and \texttt{thicklines} are self explanatory.

The package \package{easytab} provides the \envname{TAB} environment
which is a simple (re)\-implementation of the array---tabular environment,
with some limitations and some additional features.
The syntax can be either
%%
\begin{verbatim}
  \begin{TAB}`(eq)'`[ex]'`{cc...c}'`{cc...c}'
     a & b & ... & n \\
     ...
  \end{TAB}
\end{verbatim}
%%
or
%%
\begin{verbatim}
  \begin{TAB}`(eq,mx,my)'`[ex,MX,MY]'`{cc...c}'`{cc...c}'
     a & b & ... & n \\
     ...
  \end{TAB}
\end{verbatim}
%%
\begin{dotlist}
  \item
  \verb|(eq)| or \verb|(eq,mx,my)|.  By \verb|eq| you can balance
  the rows or the column or both, as shown in this table:
  %%
  \begin{center}
    \textbf{Table 1.}\nobreak\\[1em]
    \begin{tabular}{|l|l|}
      \hline
      value of \verb|eq| & effect \\
      \hline
      \verb'@' & no balancing \\
      \verb'r' & equal rows heights \\
      \verb'c' & equal column widths  \\
      \verb'b' & equal rows heights and equal column widths \\
      \verb'e' & equal rows heights and column widths \\
      \hline
    \end{tabular}
  \end{center}
  %%
  By \verb|mx| and \verb|my| you can modify the minimum size of the
  box in the \envname{TAB} environment.
  This must be a valid measure e.g. \verb|2pt|. 
  This is useful in writing matrices an vectors.
  %%
  \item
  \verb|[ex]| or \verb|[ex,MX,MY]|.  By \verb|ex| you can specify
  the amount of extra space around the item in the \envname{TAB}
  environment.  The default is \verb|2pt|.  By \verb|MX| and \verb|MY|
  you can modify the minimum size of the whole table in the 
  \envname{TAB} environment.  This must be a valid measure
  e.g.~\verb|10cm|.
  %%  
  \item
  The first \verb|{cc...c}| is the definition of the columns and
  their alignment.  The possible alignment for the columns are:
  %%
  \begin{center}
    \textbf{Table 2.} \nobreak\\[1em]
    \begin{tabular}{|l|l|}
      \hline
      \verb'c' & centering \\
      \verb'l' & flush left \\
      \verb'r' & flush right \\
      \hline
    \end{tabular}
  \end{center}
  %%
  \item
  The second \verb|{cc...c}| is the definition of the rows their
  alignment.  The possible alignment for the rows are:
  %%
  \begin{center}
    \textbf{Table 3.} \\[1em]
    \begin{tabular}{|l|l|}
      \hline
      \verb'c' & centering \\
      \verb't' & flush top \\
      \verb'b' & flush bottom \\
      \hline
    \end{tabular}
  \end{center}
  %%
\end{dotlist}
\textbf{IMPORTANT:} The package can manage matrices with a maximum 
of \verb|30| rows by \verb|30| columns.\\
%
\textbf{IMPORTANT:} The functionality of the environment ``TAB'' is 
the same of the environment \envname{BMAT} the only difference is that its 
entries are in ``text'' mode not in ``mathematic'' mode. Please read 
the documentation of the package \package{easybmat} to understand
how to use ``TAB''.

\end{document}
