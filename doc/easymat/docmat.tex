\makeatletter
\def\input@path{{../macros/}{../../macros/}}
\makeatother

\documentclass[a4paper,final,11pt]{article}

\usepackage{mydoc}
\newcommand{\package}[1]{\textsc{#1}}
\newcommand{\envname}[1]{\textsf{#1}}
\newcommand{\cmdname}[1]{\texttt{\char`\\#1}}

\author{Enrico Bertolazzi\\ \\
  Laboratorio di Matematica Applicata e Meccanica Computazionale\\
  Dipartimento di Ingegneria Meccanica e Strutturale\\
  Universit\`a degli Studi di Trento\\
  Mesiano, Trento, Italy \\ \\
  \Email{enrico.bertolazzi@ing.unitn.it} }

%%%
%%% HEADER PAGE
%%%
%begin{latexonly}
\usepackage{fancyhdr}
\fancypagestyle{fpage}{
  \fancyhf{} % clear all
  \fancyhead[el]{\fancyplain{}{\sffamily\bfseries\thepage}}
  \fancyhead[er]{\fancyplain{}{\sffamily\bfseries\leftmark}}
  \fancyhead[or]{\fancyplain{}{\sffamily\bfseries\thepage}}
  \fancyhead[ol]{\fancyplain{}{\sffamily\bfseries\rightmark}}
}
\renewcommand{\headrulewidth}{0pt}
%end{latexonly}

%%%
%%% LaTeX2HTML INCLUDES
%%%
\usepackage{html,htmllist,url,verbatim}
\bodytext{BACKGROUND = "bk.jpg"}
%begin{latexonly}
\newcommand{\URL}{\begingroup\urlstyle{sf}\Url}
\newcommand{\Email}{\begingroup\urlstyle{sf}\Url}
%end{latexonly}
\begin{htmlonly}
\newcommand{\URL}[1]{\htmladdnormallink{#1}{#1}}
\end{htmlonly}

%%
%% POSTSCRIPT AND FIGURES
%%
\usepackage{epsfig}
\graphicspath{{:figure:}{figure/}}

%begin{latexonly}
\usepackage{amssymb}
\newcommand{\CIRC}{$\circ$}
\newcommand{\BULLET}{$\bullet$}
%end{latexonly}

\begin{htmlonly}

\newcommand{\EXAMPLE}[1]{
  \begin{BOXED}
    \VRB{#1}
    \input{#1}
  \end{BOXED}
}

\newcommand{\EXAMPLEA}[1]{
  \begin{BOXED}
    \VRB{#1}
    \input{#1}
  \end{BOXED}
}

\newcommand{\EXAMPLESPLIT}[1]{
  \begin{BOXED}
    \VRB{#1.1.tex}
    \VRB{#1.2.tex}
    \input{#1.tex}
  \end{BOXED}
}

\newcommand{\FRAMECODE}[1]{
  \begin{BOXED}
    \input{#1}
  \end{BOXED}
}

\end{htmlonly}

%begin{latexonly}

\usepackage{fancyvrb}

\newdimen\MINISIZE
\newcommand{\VRB}[1]{
  \par
  \VerbatimInput[xleftmargin=\fboxsep,xrightmargin=\fboxsep]{#1}
  \par
}

\newcommand{\FRAMEVRB}[1]{
  \par
  \VerbatimInput[xleftmargin=\fboxsep,
                 xrightmargin=\fboxsep,
                 frame=single]{#1}
  \par
}

\newcommand{\FRAMECODE}[1]{
  \MINISIZE=\textwidth\relax
  \advance\MINISIZE-4\fboxsep\relax
  \begin{center}
  \par
  \fbox{%
  \begin{minipage}{\MINISIZE}
  \input{#1}
  \end{minipage}}
  \par
  \end{center}
}

\newcommand{\EXAMPLE}[1]{
  \FRAMEVRB{#1.tex}
  \FRAMECODE{#1.tex}
}

\newcommand{\EXAMPLEA}[1]{
  \MINISIZE=\textwidth\relax
  \advance\MINISIZE-4\fboxsep\relax
  \begin{center}
  \par
  \fbox{%
  \begin{minipage}{\MINISIZE}
  \vskip\fboxsep%
  \VerbatimInput[xleftmargin=\fboxsep,
                 xrightmargin=\fboxsep,
                 frame=single]{#1.tex}
  \input{#1.tex}
  \end{minipage}}
  \par
  \end{center}
}

\let\EXAMPLESPLIT\EXAMPLE

%end{latexonly}

\usepackage{easymat}%
\usepackage{easybmat}%
\usepackage[definevectors]{easyvector}
\usepackage[math]{easyeqn}

\title{The package \package{easymat}}

\begin{document}
\maketitle
\begin{abstract}
  The \package{easymat}
%     \footnote{%
%     \textbf{Important} The package was improved a lot between version
%     0.1 to version 0.2.  Unfortunately, due to some technical choice,
%     matrices written for version 0.2 are incompatible (because of few
%     modification) with matrices written for version 0.1.  If you
%     prefer the old style you can find \texttt{easymat.sty} in the
%     directory ``\texttt{old}''}
  package is a macro package for supporting block matrices
  having equal column widths or equal rows heights or both,
  and supporting various kinds of rules~(lines) between rows and columns.
  The package is based on an array/ta\-bular-like syntax.
\end{abstract}

\tableofcontents
%begin{latexonly}
\clearpage
%end{latexonly}

\pagestyle{fpage}
\def\sectionmark#1{\markboth{The package \package{easymat}}{The package \package{easymat}}}
\let\chaptermark\sectionmark
\let\subsectionmark\sectionmark

\section{Some examples with \package{easymat}}
%%
The pachage is loaded by means the usual way:
%%
\begin{verbatim}
  \documentclass{article}
  .
  .
  \usepackage[thinlines,thicklines]{easymat}
  .
  .
\end{verbatim}
%%
The options \texttt{thinlines} and \texttt{thicklines} are self explanatory.
\package{easymat} provides the \envname{MAT}~environment which is a
simple re-implementation of the
array/tabular environment, with some limitation and some additional features.
The syntax is
%%
\begin{verbatim}
  \begin{MAT}`(eq)'`[ex]'`{cc...c}'
    a & b & ... & n \\
    ...
  \end{MAT}
\end{verbatim}
%%
or
%%
\begin{verbatim}
  \begin{MAT}`(eq,mx,my)'`[ex,MX,MY]'`{cc...c}'
     a & b & ... & n \\
     ...
  \end{MAT}
\end{verbatim}
%%
\begin{dotlist}
  \item 
  \verb|(eq)| or \verb|(eq,mx,my)|.  By \verb|eq| you can balance
  the rows or the column or both, as shown in this table:
  %%
  \begin{center}
    \textbf{Table 1.} \\[1em]
    \begin{tabular}{|l|l|}
      \hline
      value of \verb|eq| & effect \\
      \hline
      \verb+@+ & no balancing \\
      \verb+r+ & equal rows heights \\
      \verb+c+ & equal column widths  \\
      \verb+b+ & equal rows heights and equal column widths \\
      \verb+e+ & equal rows heights and column widths \\
      \hline
    \end{tabular}
  \end{center}
  %%
  By \verb|mx| and \verb|my| you can modify the minimum size of the
  box in the MAT environment.  This must be a valid measure e.g.
  \verb|2pt|.  This is useful in writing matrices an vectors.
  %%
  \item
  \verb|[ex]| or \verb|[ex,MX,MY]|.  By \verb|ex| you can
  specify the amount of extra space around the item in the
  \envname{MAT} environment.  The default is \verb|2pt|.  By
  \verb|MX| and \verb|MY| you can modify the minimum size of the
  whole table in the \envname{TAB} environment.  This must be a
  valid measure e.g. \verb|10cm|.
  %%
  \item 
  The \verb|`{cc...c}'| is the definition of the columns and their
  alignment.  The possible alignment for the columns are:
  %%
  \begin{center}
    \textbf{Table 2.} \\[1em]
    \begin{tabular}{|l|l|}
      \hline
      \verb|c| & centering \\
      \verb|l| & flush left \\
      \verb|r| & flush right \\
      \hline
    \end{tabular}
  \end{center}
  %%
\end{dotlist}
%
\textbf{IMPORTANT:} The package can manage matrices with a maximum 
of \verb|30| rows by \verb|30| columns.

It is possible to produce rules among columns or rows as this 
example shows:
%%
\EXAMPLEA{docmat.1}
%%
The command \cmdname{first} is used to produce the first top rule. The 
various separation rules are defined by a character code immediately 
after the command \verb+\\+.
The available rules for the rows and columns are
%
\begin{center}
\textbf{Table 3.}\nobreak\\
\begin{tabular}{|l|l|}
\hline
nothing  & no rule \\
\verb+|+ & solid line (or \verb+-+ for the rows) \\
\verb+:+ & dash line \\
\verb+;+ & dot-dash line \\
\verb+.+ & dotted line \\
\verb+0+ & solid line with size \verb|1/5| of normal line \\
\verb+1+ & solid line with size \verb|1/4| of normal line \\
\verb+2+ & solid line with size \verb|1/3| of normal line \\
\verb+3+ & solid line with size \verb|1/2| of normal line \\
\verb+4+ & equivalent to \verb+|+ \\
\verb+5+ & solid line with size \verb|2| times of normal line \\
\verb+6+ & solid line with size \verb|3| times of normal line \\
\verb+7+ & solid line with size \verb|4| times of normal line \\
\verb+8+ & solid line with size \verb|5| times of normal line \\
\verb+9+ & solid line with size \verb|6| times of normal line \\
\hline
\end{tabular}
\end{center}

\textbf{IMPORTANT:} each row \textbf{must} end with \verb+\\+ 
otherwise an error is produced.

The main feature of the \envname{MAT}~environment
is that it is reentrant as shown below:
%%
\EXAMPLEA{docmat.2}
%%
\textbf{IMPORTANT:} The package can manage maximum reentrance 
of \verb|8| levels.


\section{Some example with balancing}
%%
The effect of various balancing is seen below:
%%
\EXAMPLEA{docmat.3}
%%
and this is another example
%%
\EXAMPLEA{docmat.4}
%%



\section{An example with minimal size setting}
%%
It is possible to specify the minimal size of the item inside a
\envname{MAT} environment:
%%
\EXAMPLEA{docmat.5}
%%
It is possible to specify the total minimal size of a \envname{MAT}
environment, as shown here
%%
\EXAMPLEA{docmat.6}
%%


\section{An example with various size rules}
%%
This example shows the use of various size rule in \envname{MAT} 
environment:
%%
\EXAMPLEA{docmat.7}
%%


\section{The \texttt{\char`\\addpath} command}
%%
Is is possible to add paths to the \envname{MAT} environment. The syntax 
is the following
%%
\begin{verbatim}

\begin{MAT} ...... {...}
   ...... \\
   ...... \\
   ...... \\
   \addpath{(`x',`y',`rule')`path'}
   .
   .
   \addpath{(`x',`y',`rule')`path'}
\end{MAT}

\end{verbatim}
%%
where
%%
\begin{desc}
  \item[x and y] are the integer coordinates of the starting
  corner.  The down left corner is at $x=0$, $y=0$.
  %%
  \item[rule] is the code of a valid rule as described in table 3.
  %%
  \item[path] is a string describing the path.  Each letter of the
  string is a movement coded as follows:
  %%
  \begin{center}
    \textbf{Table 4.} \\[1em]
    \begin{tabular}{|l|l|}
      \hline
      letter & direction \\
      \hline
      \verb|l| & left movement and drawing \\
      \verb|r| & right movement and drawing \\
      \verb|u| & up movement and drawing \\
      \verb|d| & down movement and drawing \\
      \hline
    \end{tabular}
  \end{center}
  %%
\end{desc}
%%
The following example shows the use of \cmdname{addpath},
%%
\EXAMPLEA{docmat.8}
%%
\textbf{IMPORTANT:} The commands \cmdname{addpath} must be put
\textbf{in front of} the last \verb+\\+ command.

This is another example
%%
\EXAMPLEA{docmat.9}
%%


\section{An example with reentrance}
%%
This final example shows a slightly more complex (reentrant)
definition in 
which the \envname{MAT}~environment is used:
\typeout{This example can take a while to compile on small machine, please wait...}
%%
\EXAMPLEA{docmat.10}
%%
\typeout{...thanks, I have done}

\end{document}
