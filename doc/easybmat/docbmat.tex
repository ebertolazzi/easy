\makeatletter
\def\input@path{{../macros/}{../../macros/}}
\makeatother

\documentclass[twoside,a4paper]{article}

\usepackage{mydoc}
\newcommand{\package}[1]{\textsc{#1}}
\newcommand{\envname}[1]{\textsf{#1}}
\newcommand{\cmdname}[1]{\texttt{\char`\\#1}}

\author{Enrico Bertolazzi\\ \\
  Laboratorio di Matematica Applicata e Meccanica Computazionale\\
  Dipartimento di Ingegneria Meccanica e Strutturale\\
  Universit\`a degli Studi di Trento\\
  Mesiano, Trento, Italy \\ \\
  \Email{enrico.bertolazzi@ing.unitn.it} }

%%%
%%% HEADER PAGE
%%%
%begin{latexonly}
\usepackage{fancyhdr}
\fancypagestyle{fpage}{
  \fancyhf{} % clear all
  \fancyhead[el]{\fancyplain{}{\sffamily\bfseries\thepage}}
  \fancyhead[er]{\fancyplain{}{\sffamily\bfseries\leftmark}}
  \fancyhead[or]{\fancyplain{}{\sffamily\bfseries\thepage}}
  \fancyhead[ol]{\fancyplain{}{\sffamily\bfseries\rightmark}}
}
\renewcommand{\headrulewidth}{0pt}
%end{latexonly}

%%%
%%% LaTeX2HTML INCLUDES
%%%
\usepackage{html,htmllist,url,verbatim}
\bodytext{BACKGROUND = "bk.jpg"}
%begin{latexonly}
\newcommand{\URL}{\begingroup\urlstyle{sf}\Url}
\newcommand{\Email}{\begingroup\urlstyle{sf}\Url}
%end{latexonly}
\begin{htmlonly}
\newcommand{\URL}[1]{\htmladdnormallink{#1}{#1}}
\end{htmlonly}

%%
%% POSTSCRIPT AND FIGURES
%%
\usepackage{epsfig}
\graphicspath{{:figure:}{figure/}}

%begin{latexonly}
\usepackage{amssymb}
\newcommand{\CIRC}{$\circ$}
\newcommand{\BULLET}{$\bullet$}
%end{latexonly}

\begin{htmlonly}

\newcommand{\EXAMPLE}[1]{
  \begin{BOXED}
    \VRB{#1}
    \input{#1}
  \end{BOXED}
}

\newcommand{\EXAMPLEA}[1]{
  \begin{BOXED}
    \VRB{#1}
    \input{#1}
  \end{BOXED}
}

\newcommand{\EXAMPLESPLIT}[1]{
  \begin{BOXED}
    \VRB{#1.1.tex}
    \VRB{#1.2.tex}
    \input{#1.tex}
  \end{BOXED}
}

\newcommand{\FRAMECODE}[1]{
  \begin{BOXED}
    \input{#1}
  \end{BOXED}
}

\end{htmlonly}

%begin{latexonly}

\usepackage{fancyvrb}

\newdimen\MINISIZE
\newcommand{\VRB}[1]{
  \par
  \VerbatimInput[xleftmargin=\fboxsep,xrightmargin=\fboxsep]{#1}
  \par
}

\newcommand{\FRAMEVRB}[1]{
  \par
  \VerbatimInput[xleftmargin=\fboxsep,
                 xrightmargin=\fboxsep,
                 frame=single]{#1}
  \par
}

\newcommand{\FRAMECODE}[1]{
  \MINISIZE=\textwidth\relax
  \advance\MINISIZE-4\fboxsep\relax
  \begin{center}
  \par
  \fbox{%
  \begin{minipage}{\MINISIZE}
  \input{#1}
  \end{minipage}}
  \par
  \end{center}
}

\newcommand{\EXAMPLE}[1]{
  \FRAMEVRB{#1.tex}
  \FRAMECODE{#1.tex}
}

\newcommand{\EXAMPLEA}[1]{
  \MINISIZE=\textwidth\relax
  \advance\MINISIZE-4\fboxsep\relax
  \begin{center}
  \par
  \fbox{%
  \begin{minipage}{\MINISIZE}
  \vskip\fboxsep%
  \VerbatimInput[xleftmargin=\fboxsep,
                 xrightmargin=\fboxsep,
                 frame=single]{#1.tex}
  \input{#1.tex}
  \end{minipage}}
  \par
  \end{center}
}

\let\EXAMPLESPLIT\EXAMPLE

%end{latexonly}

\usepackage{easymat}%
\usepackage{easybmat}%
\usepackage[definevectors]{easyvector}
\usepackage[math]{easyeqn}

\title{The package \package{easybmat}}

\begin{document}

\maketitle
\begin{abstract}
  The \package{easybmat} package is a macro package for writing block
  matrices, with equal column widths or equal rows heights or both,
  with various kinds of rules~(lines) between rows and columns.  It
  uses an array/ta\-bular-like syntax.
\end{abstract}


\tableofcontents
%begin{latexonly}
\clearpage
%end{latexonly}

\pagestyle{fpage}
\def\sectionmark#1{\markboth{The package \package{easybmat}}{The package \package{easybmat}}}
\let\chaptermark\sectionmark
\let\subsectionmark\sectionmark

\section{Some examples with \package{easybmat}}
Load the package in the usual way:
%
\VRB{docbmat.1.tex}
%
The options \texttt{thinlines}, and \texttt{thiklines} are self
explanatory.  \package{easybmat} provides the
\envname{BMAT}~environment which is a re-implementation of the
array/tabular environment, with some limitation and some additional
features.  The syntax is
%
\VRB{docbmat.2.tex}
%
or 
%
\VRB{docbmat.3.tex}
%
\begin{dotlist}
  %%
  \item
  %%
  \verb|(eq)| or \verb|(eq,mx,my)|.  By \verb|eq| you can balance
  the rows or the column or both, as shown in this table:
  \begin{center}
    \par
    \textbf{Table 1.}\nobreak\\[1em]
    \begin{tabular}{|l|l|}
      \hline
      value of \verb|eq| & effect \\
      \hline
      \verb|@| & no balancing \\
      \verb|r| & equal rows heights \\
      \verb|c| & equal column widths  \\
      \verb|b| & equal rows heights and equal column widths \\
      \verb|e| & equal rows heights and column widths \\
      \hline
    \end{tabular}
    \par
  \end{center}
  By \verb|mx| and \verb|my| you can modify the minimum size of the
  box in the BMAT environment.  This must be a valid measure e.g.
  2pt.  This is useful in writing matrices an vectors.
  %%
  \item
  \verb|[ex]| or \verb|[ex,MX,MY]|.  By ``\verb|ex|'' you can specify
  the amount of extra space around the item in the \envname{BMAT}
  environment.  The default is \verb|2pt|.  By \verb|MX| and \verb|MY|
  you can modify the minimum size of the whole block matrix in the
  \envname{BMAT} environment.  This must be a valid measure e.g.10cm.
  %%
  \item
  The first \verb|`{cc...c}'| is the definition of the columns and
  their alignment.  The possible alignment for the columns are:
  \begin{center}
    \par
    \textbf{Table 2.}\nobreak\\[1em]
    \begin{tabular}{|l|l|}
      \hline
      \verb|c| & centering \\
      \verb|l| & flush left \\
      \verb|r| & flush right \\
      \hline
    \end{tabular}
    \par
  \end{center}
  %%  
  \item
  The second \verb|`{cc...c}'| is the definition of the rows their
  alignment.  The possible alignment for the rows are:
  \begin{center}
    \par
    \textbf{Table 3.}\nobreak\\[1em]
    \begin{tabular}{|l|l|}
      \hline
      \verb|c| & centering \\
      \verb|t| & flush top \\
      \verb|b| & flush bottom \\
      \hline
    \end{tabular}
    \par
  \end{center}
\end{dotlist}
%
%
%
\textbf{IMPORTANT:} The package can manage matrices with a maximum of
\verb|30| rows by \verb|30| columns.

It is possible to produce rules between columns or rows as this
example shows:
%
\EXAMPLEA{docbmat.4}
%
The available rules for the rows and columns are
\begin{center}
  \par
  \textbf{Table 4.} \\[1em]
  \begin{tabular}{|l|l|}
    \hline
    nothing  & no rule \\
    \verb+|+ & solid line \\
    \verb|:| & dash line \\
    \verb|;| & dot-dash line \\
    \verb|.| & dotted line \\
    \verb|0| & solid line with size \verb|1/5| of normal line \\
    \verb|1| & solid line with size \verb|1/4| of normal line \\
    \verb|2| & solid line with size \verb|1/3| of normal line \\
    \verb|3| & solid line with size \verb|1/2| of normal line \\
    \verb|4| & equivalent to \verb+|+ \\
    \verb|5| & solid line with size \verb|2| times of normal line \\
    \verb|6| & solid line with size \verb|3| times of normal line \\
    \verb|7| & solid line with size \verb|4| times of normal line \\
    \verb|8| & solid line with size \verb|5| times of normal line \\
    \verb|9| & solid line with size \verb|6| times of normal line \\
    \hline
  \end{tabular}
  \par
\end{center}
The main feature of the \envname{BMAT}~environment
is that it is reentrant as shown here:
%
\EXAMPLEA{docbmat.5}
%
\textbf{IMPORTANT:} The package can manage a reentrance of a maximum 
of \verb|8| levels.


\latex{\clearpage}
\section{An example with balancing}
Here it is showed the effect of various balancing:
%
\EXAMPLEA{docbmat.6}
%

\latex{\clearpage}
\section{Some example with minimal size setting}
It is possible to specify the minimal size of the item inside a
``BMAT'' environment, as shown here
%
\EXAMPLEA{docbmat.7}
%
It is possible to specify the total minimal size of a ``BMAT''
environment, as shown here
%
\EXAMPLEA{docbmat.8}
%

\latex{\clearpage}
\section{An example with various size rules}
This example shows the use of various size rule in \envname{BMAT} 
environment:
%
\EXAMPLEA{docbmat.9}
%


\section{The \texttt{\char`\\addpath} command}
Is is possible to add paths to the ``BMAT'' environment. The syntax 
is the following
%
\VRB{docbmat.10.tex}
%
where
\begin{desc}
  %%
  \item["x' and `y']
  are the integer coordinate of the starting point.  The left down
  corner is at coordinate $x=0$, $y=0$.
  %%
  \item[rule]
  is the code of a valid rule as described in table 4.
  %%
  \item[path]
  is a string describing the path.  Each letter of the string is a
  movement coded as follows:
  \begin{center}
    \par
    \textbf{Table 5.} \\[1em]
    \begin{tabular}{|l|l|}
      \hline
      letter & direction \\
      \hline
      \verb|l| & left movement and drawing \\
      \verb|r| & right movement and drawing \\
      \verb|u| & up movement and drawing \\
      \verb|d| & down movement and drawing \\
      \hline
    \end{tabular}
    \par
  \end{center}
\end{desc}
%
The following example shows the use of \cmdname{addpath},
%
\EXAMPLEA{docbmat.11}
%
This is another example
%
\EXAMPLEA{docbmat.12}
%


\section{An example with reentrance}
This final example shows a slightly more complex (reentrant)
definition in which the \envname{BMAT}~environment is used:
%
\VRB{docbmat.13.tex}
%
It produces the following output:
\typeout{This example can take a while to compile on small machine, please wait...}
%
\FRAMECODE{docbmat.13.tex}
%
\typeout{...thanks, I have done}

\end{document}
