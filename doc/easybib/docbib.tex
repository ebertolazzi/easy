\makeatletter
\def\input@path{{../macros/}{../../macros/}}
\makeatother

\documentclass[a4paper,final,11pt]{article}

\usepackage{mydoc}
\newcommand{\package}[1]{\textsc{#1}}
\newcommand{\envname}[1]{\textsf{#1}}
\newcommand{\cmdname}[1]{\texttt{\char`\\#1}}

\author{Enrico Bertolazzi\\ \\
  Laboratorio di Matematica Applicata e Meccanica Computazionale\\
  Dipartimento di Ingegneria Meccanica e Strutturale\\
  Universit\`a degli Studi di Trento\\
  Mesiano, Trento, Italy \\ \\
  \Email{enrico.bertolazzi@ing.unitn.it} }

%%%
%%% HEADER PAGE
%%%
%begin{latexonly}
\usepackage{fancyhdr}
\fancypagestyle{fpage}{
  \fancyhf{} % clear all
  \fancyhead[el]{\fancyplain{}{\sffamily\bfseries\thepage}}
  \fancyhead[er]{\fancyplain{}{\sffamily\bfseries\leftmark}}
  \fancyhead[or]{\fancyplain{}{\sffamily\bfseries\thepage}}
  \fancyhead[ol]{\fancyplain{}{\sffamily\bfseries\rightmark}}
}
\renewcommand{\headrulewidth}{0pt}
%end{latexonly}

%%%
%%% LaTeX2HTML INCLUDES
%%%
\usepackage{html,htmllist,url,verbatim}
\bodytext{BACKGROUND = "bk.jpg"}
%begin{latexonly}
\newcommand{\URL}{\begingroup\urlstyle{sf}\Url}
\newcommand{\Email}{\begingroup\urlstyle{sf}\Url}
%end{latexonly}
\begin{htmlonly}
\newcommand{\URL}[1]{\htmladdnormallink{#1}{#1}}
\end{htmlonly}

%%
%% POSTSCRIPT AND FIGURES
%%
\usepackage{epsfig}
\graphicspath{{:figure:}{figure/}}

%begin{latexonly}
\usepackage{amssymb}
\newcommand{\CIRC}{$\circ$}
\newcommand{\BULLET}{$\bullet$}
%end{latexonly}

\begin{htmlonly}

\newcommand{\EXAMPLE}[1]{
  \begin{BOXED}
    \VRB{#1}
    \input{#1}
  \end{BOXED}
}

\newcommand{\EXAMPLEA}[1]{
  \begin{BOXED}
    \VRB{#1}
    \input{#1}
  \end{BOXED}
}

\newcommand{\EXAMPLESPLIT}[1]{
  \begin{BOXED}
    \VRB{#1.1.tex}
    \VRB{#1.2.tex}
    \input{#1.tex}
  \end{BOXED}
}

\newcommand{\FRAMECODE}[1]{
  \begin{BOXED}
    \input{#1}
  \end{BOXED}
}

\end{htmlonly}

%begin{latexonly}

\usepackage{fancyvrb}

\newdimen\MINISIZE
\newcommand{\VRB}[1]{
  \par
  \VerbatimInput[xleftmargin=\fboxsep,xrightmargin=\fboxsep]{#1}
  \par
}

\newcommand{\FRAMEVRB}[1]{
  \par
  \VerbatimInput[xleftmargin=\fboxsep,
                 xrightmargin=\fboxsep,
                 frame=single]{#1}
  \par
}

\newcommand{\FRAMECODE}[1]{
  \MINISIZE=\textwidth\relax
  \advance\MINISIZE-4\fboxsep\relax
  \begin{center}
  \par
  \fbox{%
  \begin{minipage}{\MINISIZE}
  \input{#1}
  \end{minipage}}
  \par
  \end{center}
}

\newcommand{\EXAMPLE}[1]{
  \FRAMEVRB{#1.tex}
  \FRAMECODE{#1.tex}
}

\newcommand{\EXAMPLEA}[1]{
  \MINISIZE=\textwidth\relax
  \advance\MINISIZE-4\fboxsep\relax
  \begin{center}
  \par
  \fbox{%
  \begin{minipage}{\MINISIZE}
  \vskip\fboxsep%
  \VerbatimInput[xleftmargin=\fboxsep,
                 xrightmargin=\fboxsep,
                 frame=single]{#1.tex}
  \input{#1.tex}
  \end{minipage}}
  \par
  \end{center}
}

\let\EXAMPLESPLIT\EXAMPLE

%end{latexonly}

\usepackage{easymat}%
\usepackage{easybmat}%
\usepackage[definevectors]{easyvector}
\usepackage[math]{easyeqn}

\usepackage[definethebibliography]{easybib}

%begin{latexonly}
\newcommand{\AmSTeX}{{$\mathcal{A}$\kern-.1667em 
\lower.5ex\hbox{$\mathcal{M}$}\kern-.125em$\mathcal{S}$}-\TeX}
%end{latexonly}

\title{The package \package{easybib}}

\begin{document}

\maketitle
\begin{abstract}
  The package \package{easybib}
  introduces new items for easy custom-made bibliographies.
\end{abstract}

\tableofcontents

%begin{latexonly}
\vfill
\clearpage
%end{latexonly}

\pagestyle{fpage}
\def\sectionmark#1{\markboth{The package \package{easybib}}{The package \package{easybib}}}
\let\chaptermark\sectionmark
\let\subsectionmark\sectionmark

\section{The package \package{easybib}}
%%
The scheme of the bibliography in the package \package{easybib}
was inspired by the bibliography system of \AmSTeX{}.
%%
For the sake of flexibility, the syntax is somewhat different,
yet the functionality is similar (although not identical).
For the use, load the package using the usual syntax:
%%
\begin{verbatim}
  
  \documentclass{article}
  .
  .
  \usepackage[definethebibliography]{easybib}
  .
  .

\end{verbatim}
%%
The option \cmdname{definethebibliography} overrides
the default \envname{thebibliography} environment.
In this case the environment \envname{thebibliography}
takes the form:
%%
\begin{verbatim}
  
  \begin{thebibliography}``[optional name]''
                         ``(\cmd,space)''{99}
  .
  .
  .
  \end{thebibliography}

\end{verbatim}
%%
so that you can override the default name \cmdname{refname}
with the name of your choice, you can change the default \cmdname{section*} with
\cmdname{cmd} and add extra space \verb'space' in front of the items.  The
syntax of the bibliography command is now the following
%%
\FRAMEVRB{docbib.1.tex}
%%
the command
%%
\begin{verbatim}

  \bookref``[display label]''{label} ... \endref

\end{verbatim}
%%
is used to refer to a book, while 
%%
\begin{verbatim}

  \paperref``[display label]''{label} ... \endref

\end{verbatim}
%%
is used to refer to a paper.

When an item, for example \cmdname{xxx}, is encountered then the
following text is expanded as follows
%%
\begin{verbatim}

  \xxx+text ==> punctation + 
                begin commands + 
		text + 
		end commands

\end{verbatim}
%%
where the \verb|punctation| is displayed unless \cmdname{xxx} is the 
first displayed item.
The default values for the items in the \cmdname{bookref} environment 
are the following
%%
\begin{center}
\par
\begin{tabular}{|l|l|l|l|}
\hline
command & punctation  & begin commands & end commands \\
\hline
\cmdname{by}      & \verb',' & \cmdname{bfseries} & nothing \\
\cmdname{bysame}  & \verb',' &
   $\vcenter{\vskip.5em\hbox{\verb'\hbox to3em'}\vskip.2em
    \hbox{\verb'{\hrulefill\hskip.1em}'}\vskip.5em}$ & nothing \\
\cmdname{title}     & \verb',' & \cmdname{scshape} & nothing \\
\cmdname{bookinfo}  & \verb',' & \cmdname{rmfamily} & nothing \\
\cmdname{publ}      & \verb',' & \cmdname{rmfamily} & nothing \\
\cmdname{publaddr}  & \verb',' & \cmdname{rmfamily} & nothing \\
\cmdname{pages}     & \verb',' & \cmdname{rmfamily} & nothing \\
\cmdname{yr}        & \verb',' & \cmdname{rmfamily} & nothing \\
\cmdname{lang}      & nothing  & \verb'(' & \verb')' \\
\cmdname{transl}    & \verb',' & \cmdname{rmfamily} & nothing \\
\hline
\end{tabular}
\par
\end{center}
%%
The default values for the items in the \cmdname{paperref} environment
are the following
%%
\begin{center}
\par
\begin{tabular}{|l|l|l|l|}
\hline
command & punctation  & begin commands & end commands \\
\hline
\cmdname{by}       & \verb',' & \cmdname{bfseries} & nothing \\
\cmdname{bysame}   & \verb',' &
   $\vcenter{\vskip.5em\hbox{\verb'\hbox to3em'}\vskip.2em
    \hbox{\verb'{\hrulefill'}\vskip.2em
    \hbox{\verb' \hskip.1em}'}\vskip.5em}   
    $ & nothing \\
\cmdname{title}     & \verb',' & \cmdname{itshape} & nothing \\
\cmdname{transl}    & \verb',' & \cmdname{rmfamily} & nothing \\
\cmdname{jour}      & \verb',' & \cmdname{rmfamily} & nothing \\
\cmdname{toappear}  & nothing  & \verb'(to appear' & \verb')' \\
\cmdname{inbook}    & \verb',' & \cmdname{rmfamily} & nothing \\
\cmdname{publ}      & \verb',' & \cmdname{rmfamily} & nothing \\
\cmdname{eds}       & nothing  & \verb'(' & \verb'\@killglue, eds.)' \\
\cmdname{publaddr}  & \verb',' & \cmdname{rmfamily} & nothing \\
\cmdname{vol}       & nothing  & \cmdname{bfseries} & nothing \\
\cmdname{yr}        & nothing  & \verb'(' & \verb')' \\
\cmdname{pages}     & \verb',' & \cmdname{rmfamily} & nothing \\
\cmdname{finalinfo} & \verb',' & \cmdname{rmfamily} & nothing \\
\cmdname{lang}      & nothing  & \verb'(' & \verb')' \\
\hline
\end{tabular}
\par
\end{center}




\section{The command \cmdname{moreref}}
%%
In the case of more than one reference of the same author or in the 
case of a series of papers or books on the same argument, it may be
useful to use the \cmdname{moreref} command. The syntax is the following
%%
\begin{verbatim}

  \paperref{label name} or \bookref{label name}
  items
  \moreref`[punctation]'{book or paper}
  items
  \moreref`[punctation]'{book or paper}
  .
  .    
  \endref

\end{verbatim}
%%
The effect is to use a single label reference for more than one book or
paper. The optional command ``\verb'[punctation]''' can be used to 
change the default punctation ``\verb';''' to something else, for example
you can use \cmdname{moreref}\verb'[, see also:]{book}'.


\section{The command \cmdname{endref}}
%%
The command \cmdname{endref} closes the definition of a reference.
By default the reference is closed by the semicolum \verb|;|.
It is possible to change the default value as follows
%%
\begin{verbatim}

  \endref[punctation], for example \endref[.]

\end{verbatim}
%%
This is useful for the last reference, for example
%%
\EXAMPLE{docbib.2}
%%


\section{The command \texttt{\char`\\refstyle}}
%%
There are many styles for display the labels of the bibliography. The
standard \LaTeX{} way of changing the appearance is to modify the
\cmdname{@bibitem} macro. A easiest way in \package{easybib} is to use
\cmdname{refstyle} before \verb'\begin{thebibliography}'.
The syntax is the following:
%%
\begin{verbatim}

  \refstyle{A} or \refstyle{B}  or  \refstyle{C}

\end{verbatim}
%%
the effect is to change \cmdname{@bibitem} as follows
%%
\begin{center}
\par
\begin{tabular}{|l|l|c|}
   \hline
   command & \cmdname{@bibitem} definition & sample output \\
   \hline
   \cmdname{refstyle}\verb'{A}' & \verb'\def\@bibitem#1{#1.}'  & 1.   \\
   \cmdname{refstyle}\verb'{B}' & \verb'\def\@bibitem#1{[#1]}' & [1]  \\
   \cmdname{refstyle}\verb'{C}' & \verb'\def\@bibitem#1{}'     & nothing \\
   \hline
\end{tabular}
\par
\end{center}




\section{An example of bibliography from \AmSTeX{} documentation}
%%
The following complex example shows the features of \envname{easybib}
and is essentially the example showed in the \AmSTeX{} documentation
translated in the language of \envname{easybib}.
%%
\EXAMPLESPLIT{docbib.3}
%%
\EXAMPLE{docbib.4}
%%


\section{Modifying the appearance}
%%
You can control the default formats by the command
%%
\begin{verbatim}

  \bibsetfmt[group name,item]{punctation}
                             {begin commands}
                             {end commands}

\end{verbatim}
%%
for example
%%
\begin{verbatim}

  \bibsetfmt[paper,by]{,}{\textit}{:}

\end{verbatim}
%%
This way you can easily modify the appearance of the
bibliography.

\section{Changing the order of the items}
%%
The order of the items:
%%
\begin{verbatim}

  for `paper' group:
  by,bysame,title,transl,jour,toappear,
  inbook,publ,eds,publaddr,vol,
  yr,pages,finalinfo,lang
  
  for `book' group:
  by,bysame,title,bookinfo,publ,publaddr,
  pages,yr,lang,transl

\end{verbatim}
%%
can be changed defining the macro \cmdname{paperlist} and
\cmdname{booklist}. For example to have the \verb|yr| item 
displayed after \verb|pages| item in the \verb|paper| group define:
%%
\begin{verbatim}

  \def\paperlist{by,bysame,title,transl,jour,%
                 toappear,inbook,publ,eds,%
		 publaddr,vol,pages,yr,%
		 finalinfo,lang}

\end{verbatim}


\section{Citing}
%%
The following segment of code
%%
\EXAMPLE{docbib.5}
%%
for the previous bibliography
%%

\section{Defining new styles}
%%
If you do not like the predefined styles \verb'paper' and \verb'book'
you can easily define new ones. For example suppose you want to 
define a new style \verb'tales' with the item \cmdname{author},
\cmdname{title}, \cmdname{year} with:
%%
\begin{desc}
  \item[\cmdname{author}] in smallcaps style surrounded by a box
  \item[\cmdname{title}] in italic style
  \item[\cmdname{year}]  in bold style  within \verb|(...)|
\end{desc}
%%
you must follow the following steps
%%
\begin{desc}
  \item[define the list of items]
%%
\begin{verbatim}

  \def\taleslist{author,title,year}

\end{verbatim}
%%
  \item[define the command \cmdname{talesref} with the items]
%%
\begin{verbatim}

  \bibdefinestyles{tales}

\end{verbatim}
%%
  it defines the items \cmdname{author}, \cmdname{title},
  \cmdname{year} with the default format and the command
  \cmdname{talesref}.
  \item[change the default formatting]
%%
\begin{verbatim}

  \bibsetfmt[tales,author]
    {,}
    {\setbox0\hbox\bgroup\scshape}
    {\egroup\fbox{\box0}}
  \bibsetfmt[tales,title]{}{\itshape}{}
  \bibsetfmt[tales,year]{}{\bfseries(}{)}

\end{verbatim}
%%
\end{desc}
%
The following example (which uses [definethebibliography])
shows the effect:
%
\EXAMPLE{docbib.6}
%

\end{document}
